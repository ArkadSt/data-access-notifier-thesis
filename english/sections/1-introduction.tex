\section{Introduction} \label{Introduction}

In Estonia there are a lot of state databases holding users' data, like Population Registry (\textit{Rahvastikuregister}) and Health Portal (\textit{Terviseportaal}). People's data in these databases is accessed by different parties for the variety of purposes. Usually those are legitimate purposes, like a doctor accessing person's health data, or even people themselves accessing their data in some information systems. However, sometimes the purpose of data access is not clear.

For the purposes of making the process more transparent, the Estonian Information System Authority (RIA) created a special service, Data Tracker (\textit{Andmejälgija}), in 2017, through which users can check which parties have accessed their data. \cite{err-population-registry-unauthorized-access}

The Data Tracker is a people-oriented service on the state portal eesti.ee, which aims to ensure transparency in the processing of personal data in the public sector. The data tracker relies on the ability of each data repository to store the data processing taking place within itself in the form of logs, in order to later display it to the individual, i.e. the data subject, via the service on eesti.ee.\cite{aj-github}

Architecturally, it is a fully distributed system, i.e. the information displayed to the user comes directly from the database that implemented the Data Tracker service. At the user's request, eesti.ee makes a query to each of the Data Tracker services and displays the query response without saving it.\cite{aj-github}

The Data Tracker should display to the individual information about data processing taking place locally in the database (activities of officials-employees with personal data) as well as an overview of when data has been transferred to a third party (via X-Road to another government agency, company, etc.).\cite{aj-github}

The Data Tracker doesn't notify, however, when the data is accessed by someone. In order to learn about the update in the data access logs, the person has to go to the eesti.ee web-view and manually query access logs from specific databases. 

The primary objective of this thesis is to solve this problem by creating a mobile phone app that would notify it's users about near-real time updates in the data access logs. 

Additionally I would like to examine existing state databases, including whether they provide access logs or not.